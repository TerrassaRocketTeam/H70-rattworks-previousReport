% !TEX root = ../main.tex

\section*{Safety concerns}

In this section some of the safety concerns of the static launch will be addressed. If any of these concerns fail to pass during the test, this will have to be postponed until they are solved.

As previously mentioned one of the safety concerns is related with the temperature and the pressure inside the tanks. As stated, the plumbing system can only safely withstand 70 bars. In order for this to happen, the gas -ergo, the tank- must be at less than 34.7ºC; any higher temperature will overtake the limit pressure.

Another safety concern has to do with the risk of explosion of the combustion chamber. The main reason why the chamber would explode is if nozzle gets obstructed and the pressure in the chamber starts climbing. This motor however, does not have the fuel and the oxidizer in the same chamber like a solid rocket does. For this reason, when the pressure is higher than the pressure in the oxidizer tank, no more oxidizer is let into the chamber and the combustion stops. However, if the pressure in the combustion chamber climbs to fast, the security valve of the oxidizer tank would kick in and start venting in order to lower the total pressure inside the oxidizer tank and the combustion chamber.

For further security, both the $N_2O$ tank valve and the electro-valve will be closed when performing the test.

The test location has also been carefully selected. The test will be performed on a stainless steel bench, anchored on a concrete ground. Away from inflammable materials or people. Also, the motor itself will be covered with stainless steel mesh.
